\subsection{Saturated star fitting and removal}
\label{sec:satstar}
\textcolor{red}{We might remove this from Paper I because we don't need it for ice stuff.}
We attempted to remove the PSF around saturated stars, and measure these stars' brightness, by selecting on saturated pixels and fitting PSF models.
We used the \texttt{webbpsf} package to obtain models for each module.
We then used \texttt{photutils} for photometry.
The saturated star fitting is implemented in the script \texttt{saturated\_star\_finding.py} in the analysis repository\footnote{\url{https://github.com/keflavich/brick-jwst-2221}}.

To identify the saturated stars, we created a custom star finder.
The finder treats all pixels with value equal to zero as `saturated'.
It excludes pixels within a specified threshold (default 50) of the edge of the image.
For each star, it reports the center as the centroid of the saturated pixels after assigning each pixel a value of 1.
It excludes any objects with a contiguous number of saturated pixels above a specified threshold (default 100) assuming such regions are not stars.
Finally, it checks if the object is star-like by examining the surrounding pixels and checking that the total value of the pixels within 3 pixels in any direction of the saturated pixels is $>500$ MJy/sr.
This last check should hold for any saturated stars, but it will reject pixels that are zero because they were never covered by a `good' pixel.
Optionally, this check will also verify that the pixels within 2 pixels of the saturated center are brighter than those further than 2 but within 4 pixels; this checks that the region around the saturated pixels is still star-like.

The saturated stars are then iteratively removed.
We remove the most saturated stars first, followed by those with progressively fewer pixels.
This iterative approach is done because the saturated stars require very large PSF footprints and therefore are very slow;
the most saturated stars can affect regions $>100$ pixels away from the saturated pixels.
In each iteration, we use photutils' \texttt{BasicPSFPhotometry} with a group maker \texttt{DAOGroup} using a separation of 8 pixels.
In each iteration, we subtract the best-fit PSF and save the resulting image.
The final image is then used for further photometry.