The HII region G000.208-0.003 is a prominent ``elephant trunk'' pillar in which an (O-star?  TBD!) has recently broken out of the pillar head and started producing a hemispherical HII region.
The green haze in Figure \ref{fig:spitzerinsetzooms}, which comes from the F466N filter, also appears in the F410M filter, and therefore it is likely to be scattered continuum emission.
The whole bubble is filled with emission at F410M and F466N as well, but the inner bubble is substantially brighter in F405N.
These features are the signature of a dense HII region that is producing substantial free-free emission in the medium bands.
This source was also mapped in \citet{Wang2010} with HST in Pa$\alpha$ and in \citet{Lu2019a} it is their source M17.
\todo{It was observed with Spitzer's IRS spectrograph but has not been discussed in detail in any papers; \citet{An2011} excluded it from their analysis, iiuc.}
 \subsection{Gas Morphology}
 \todo{Not sure this section deserves a place, it's 0\% quantitative.  But.... the brick looks so much cooler now!}
 Figure \ref{fig:fullfieldstarless} reveals a wispy edge to The Brick.
 This figure highights the emission in the Br$\alpha$ line in orange and, most likely, reflection nebulosity in the F466N filter in blue (it is possible gas-phase CO v=1-0 emission contributes, but we do not expect any gas to be both hot and dense enough to produce such emission without destroying the CO).
 The dark regions that dominate the image are caused by extinction from the dust in The Brick.
 
 In the bottom-left\footnote{We refrain from using cardinal directions because the JWST images are most conveniently presented in a frame that is aligned neither with Galactic nor celestial coordinates.} section of The Brick, we see elongated, streaked dark lanes reminiscent of wet bread dough.
 Similar structured, streaky features can be seen throughout the image, but they do not share a common orientation or origin.
 While such stretched features suggest gravitational collapse or shear, there is no clear dominant mode affecting all of the extinction features.
 The three small, isolated dark clouds in the bottom-left edge show the greatest contrast, possibly hinting that they are somewhat further in the foreground.

 In the same region, but behind the dark lanes, there appears to be an interconnected network of emission filaments.
 Many overlapping linear and curved features are apparent in both Br$\alpha$ and F466N, indicating that these are reflection features and not direct emission; they are highlighting some outer cloud layers.

 By contrast, the top-right of the image shows asymmetric illuminated features that point toward an illuminating source in the direction of the brick.
 There are elongated, fairly sharp edges seen in emission or reflection on the right side.
 
The relative dearth of sources in the center of The Brick (or the relative excess along its edge) will be partly remedied with deeper wide-band data.
Once the presence of a source is known within the center of the Brick, upper limits in the F466N filter will put useful constraints on the depth of the CO ice features.

\begin{enumerate} % Mar 25, 2023
    \item F466N blueing (CO ice) is correlated with extinction (measured from F212N/F410M)
    \item How does integrated intensity of CO correlate with dust emission?  Dust column?  Dust temperature?
    \item There are more blue stars in front of the brick, fewer red stars: selection biases.  Cataloger biases?
\end{enumerate}

\subsection{Features of Note}
Figure \ref{fig:spitzerinsetzooms} shows zoom-in frames highlighting some of the more prominent features in the JWST data.



The lower-center source in Figure \ref{fig:spitzerinsetzooms} is source C36 in \citet{Lu2019a}, AKA [RZ2013] JVLA 1 AKA 
 [IMS2012] B.
 Unlike the other HII region, this one exhibits no emission in the other filters, indicating that the emission is exclusively Br$\alpha$ with little to no continuum free-free emission or reflected continuum.
 The morphology of this HII region loosely resembles the ring-lobe nebula SBW1 \citep{Smith2013}, possibly suggesting that the nebula is comprised of ejecta from a central source.
 However, unlike that example, there is no clear central source; the crowded field makes it difficult to assess which star is responsible for ionizing this nebula.
 \todo{Also observed by Spitzer IRS.}
